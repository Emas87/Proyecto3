\documentclass[xcolor=table]{beamer}

\mode<presentation> {

%\usetheme{default}
%\usetheme{AnnArbor}
%\usetheme{Antibes}
%\usetheme{Bergen}
%\usetheme{Berkeley}
%\usetheme{Berlin}
%\usetheme{Boadilla}
%\usetheme{CambridgeUS}
%\usetheme{Copenhagen}
%\usetheme{Darmstadt}
%\usetheme{Dresden}
%\usetheme{Frankfurt}
%\usetheme{Goettingen}
%\usetheme{Hannover}
%\usetheme{Ilmenau}
%\usetheme{JuanLesPins}
%\usetheme{Luebeck}
\usetheme{Madrid}
%\usetheme{Malmoe}
%\usetheme{Marburg}
%\usetheme{Montpellier}
%\usetheme{PaloAlto}
%\usetheme{Pittsburgh}
%\usetheme{Rochester}
%\usetheme{Singapore}
%\usetheme{Szeged}
%\usetheme{Warsaw}

%\usecolortheme{albatross}
%\usecolortheme{beaver}
%\usecolortheme{beetle}
%\usecolortheme{crane}
%\usecolortheme{dolphin}
%\usecolortheme{dove}
%\usecolortheme{fly}
%\usecolortheme{lily}
%\usecolortheme{orchid}
%\usecolortheme{rose}
%\usecolortheme{seagull}
%\usecolortheme{seahorse}
%\usecolortheme{whale}
%\usecolortheme{wolverine}

\setbeamertemplate{footline} % To remove the footer line in all slides uncomment this line
%\setbeamertemplate{footline}[page number] % To replace the footer line in all slides with a simple slide count uncomment this line

%\setbeamertemplate{navigation symbols}{} % To remove the navigation symbols from the bottom of all slides uncomment this line
}

\renewcommand{\tablename}{Tabla}

\usepackage{graphicx} % Allows including images
\usepackage{booktabs} % Allows the use of \toprule, \midrule and \bottomrule in tables
\usepackage{xcolor}

%----------------------------------------------------------------------------------------
%	TITLE PAGE
%----------------------------------------------------------------------------------------

\title[Proyecto 3]{Proyecto 3: Scheduling en Tiempo Real} % The short title appears at the bottom of every slide, the full title is only on the title page

\author{Oscar Blandino, Emmanuel Barrantes, Esteban Chavarria} % Your name
\institute[TEC] % Your institution as it will appear on the bottom of every slide, may be shorthand to save space
{
Tecnologico de Costa Rica \\ % Your institution for the title page
Sistemas Operativos Avanzados \\
I Semestre - 2018
\medskip
\textit{} % Your email address
}
\date{\today} % Date, can be changed to a custom date

\begin{document}

\begin{frame}
\titlepage % Print the title page as the first slide
\end{frame}

%\begin{frame}
%\frametitle{Overview} % Table of contents slide, comment this block out to remove it
%\tableofcontents % Throughout your presentation, if you choose to use \section{} and \subsection{} commands, these will automatically be printed on this slide as an overview of your presentation
%\end{frame}

%----------------------------------------------------------------------------------------
% BEGIN OF PRESENTATION SLIDES
%----------------------------------------------------------------------------------------
%------------------------------------------------

\section{Rate Monotonic}

\subsection{Algoritmo RM}

\begin{frame} 
\frametitle{Algoritmo Rate Monotonic (RM)} 
Explicacion Basica de RM \\
\end{frame}

%------------------------------------------------
\subsection{Tests de Schedulability  Rate Monotonic } 

%------------------------------------------------
\begin{frame} 
\frametitle{Tests de Schedulability  Rate Monotonic } 
\begin{table} 
\centering 
\begin{tabular}{|l|l|l|} 
\hline 
Tareas & Periodo & T. Ejecucion \\ \hline 
T1   & 3  &  1\\ \hline 
T2   & 5  &  2\\ \hline 
T3   & 4  &  2\\ \hline 
\end{tabular} 
\caption{Datos  Rate Monotonic } 
\end{table} 
Mu =  1,233333 \\ 
U(n) =  0,779763 \\ 
\end{frame} 

%------------------------------------------------
\subsection{Tabla de Tiempo  Rate Monotonic } 

%------------------------------------------------
\begin{frame} 
\frametitle{Tabla de Tiempo  Rate Monotonic } 
\begin{table} 
\centering 
\begin{tabular}{|l|l|l|l|l|l|l|} 
\hline 
St & \cellcolor{green} & \cellcolor{green} & \cellcolor{green} & \cellcolor{green} & \cellcolor{green} & \cellcolor{red} \\ \hline 
T1 & \cellcolor{blue} & & & \cellcolor{blue} & & \\ \hline 
T2 & & & & & & \\ \hline 
T3 & & \cellcolor{cyan} & \cellcolor{cyan} & & \cellcolor{cyan} & \cellcolor{cyan} \\ \hline 
\end{tabular} 
\caption{ Rate Monotonic 1 } 
\end{table} 
Escala Bloque : Ciclos = 1 : 1 \\ 
Posicion Fallo:  5 \\ 
mcm:  60 \\ 
\end{frame} 

%------------------------------------------------
%------------------------------------------------

\section{Earliest Dead First}

\subsection{Algoritmo EDF}

\begin{frame} 
\frametitle{Algoritmo Earliest Dead First (EDF)} 
Explicacion Basica de EDF \\
\end{frame}

%------------------------------------------------
\subsection{Tests de Schedulability  Earliest Dead First } 

%------------------------------------------------
\begin{frame} 
\frametitle{Tests de Schedulability  Earliest Dead First } 
\begin{table} 
\centering 
\begin{tabular}{|l|l|l|} 
\hline 
<<<<<<< HEAD
Tareas & Periodo & T. Ejecucion \\ \hline 
T1   & 3  &  1\\ \hline 
T2   & 5  &  2\\ \hline 
T3   & 4  &  2\\ \hline 
=======
\cellcolor{lightgray}Tarea & \cellcolor{lightgray}$P_i$ & \cellcolor{lightgray}$C_i$ \\ \hline 
T1   & 6  &  1\\ \hline 
T2   & 6  &  2\\ \hline 
T3   & 6  &  3\\ \hline 
T4   & 6  &  4\\ \hline 
T5   & 6  &  5\\ \hline 
T6   & 6  &  6\\ \hline 
>>>>>>> fd36befdf5b77500bc70c153b608ea71d8d72bbf
\end{tabular} 
\caption{Datos  Earliest Dead First } 
\end{table} 
<<<<<<< HEAD
Mu =  1,233333 \\ 
U(n) =  0,779763 \\ 
=======
Condicion: $\mu \leq U(n)$ \\ 
$\mu =  3,500000 $ \\ 
$U(n) =  0,734772 $ \\ 
Dado que $\mu>U(n)$ el algoritmo indica que las tareas no son schedulable \\ 
>>>>>>> fd36befdf5b77500bc70c153b608ea71d8d72bbf
\end{frame} 

%------------------------------------------------
\subsection{Tabla de Tiempo  Earliest Dead First } 

%------------------------------------------------
\begin{frame} 
\frametitle{Tabla de Tiempo  Earliest Dead First } 
\begin{table} 
\centering 
\begin{tabular}{|l|l|l|l|l|l|l|l|l|l|l|l|} 
\hline 
<<<<<<< HEAD
St & \cellcolor{green} & \cellcolor{green} & \cellcolor{green} & \cellcolor{green} & \cellcolor{green} & \cellcolor{green} & \cellcolor{green} & \cellcolor{green} & \cellcolor{green} & \cellcolor{green} & \cellcolor{red} \\ \hline 
T1 & \cellcolor{blue} & & & & & \cellcolor{blue} & & & \cellcolor{blue} & & \\ \hline 
T2 & & & & \cellcolor{purple} & \cellcolor{purple} & & & & & \cellcolor{purple} & \cellcolor{purple} \\ \hline 
T3 & & \cellcolor{cyan} & \cellcolor{cyan} & & & & \cellcolor{cyan} & \cellcolor{cyan} & & & \\ \hline 
=======
\cellcolor{lightgray}Tarea & \cellcolor{lightgray}$P_i$ & \cellcolor{lightgray}$C_i$ \\ \hline 
T1   & 6  &  1\\ \hline 
T2   & 6  &  2\\ \hline 
T3   & 6  &  3\\ \hline 
T4   & 6  &  4\\ \hline 
T5   & 6  &  5\\ \hline 
T6   & 6  &  6\\ \hline 
>>>>>>> fd36befdf5b77500bc70c153b608ea71d8d72bbf
\end{tabular} 
\caption{ Earliest Dead First 1 } 
\end{table} 
<<<<<<< HEAD
Escala Bloque : Ciclos = 1 : 1 \\ 
Posicion Fallo:  10 \\ 
mcm:  60 \\ 
=======
Condicion: $\mu \leq 1$ \\ 
$\mu =  3,500000 $ \\ 
Dado que $\mu>1$ el algoritmo indica que las tareas no son schedulable \\ 
>>>>>>> fd36befdf5b77500bc70c153b608ea71d8d72bbf
\end{frame} 

%------------------------------------------------
%------------------------------------------------

\section{Least Laxity First}

\subsection{Algoritmo LLF}

\begin{frame} 
\frametitle{Algoritmo Least Laxity First (LLF)} 
Explicacion Basica de LLF \\
\end{frame}

%------------------------------------------------
\subsection{Tests de Schedulability  Least Laxity First } 

%------------------------------------------------
\begin{frame} 
\frametitle{Tests de Schedulability  Least Laxity First } 
\begin{table} 
\centering 
\begin{tabular}{|l|l|l|} 
\hline 
<<<<<<< HEAD
Tareas & Periodo & T. Ejecucion \\ \hline 
T1   & 3  &  1\\ \hline 
T2   & 5  &  2\\ \hline 
T3   & 4  &  2\\ \hline 
\end{tabular} 
\caption{Datos  Least Laxity First } 
\end{table} 
Mu =  1,233333 \\ 
U(n) =  0,779763 \\ 
=======
\cellcolor{lightgray}Tarea & \cellcolor{lightgray}$P_i$ & \cellcolor{lightgray}$C_i$ \\ \hline 
T1   & 6  &  1\\ \hline 
T2   & 6  &  2\\ \hline 
T3   & 6  &  3\\ \hline 
T4   & 6  &  4\\ \hline 
T5   & 6  &  5\\ \hline 
T6   & 6  &  6\\ \hline 
\end{tabular} 
\caption{Datos  Least Laxity First } 
\end{table} 
$\mu =  3,500000 $ \\ 
$U(n) =  0,734772 $ \\ 
>>>>>>> fd36befdf5b77500bc70c153b608ea71d8d72bbf
\end{frame} 

%------------------------------------------------
\subsection{Tabla de Tiempo  Least Laxity First } 

%------------------------------------------------
\begin{frame} 
\frametitle{Tabla de Tiempo  Least Laxity First } 
\begin{table} 
\centering 
<<<<<<< HEAD
\begin{tabular}{|l|l|l|l|l|l|l|l|l|l|l|} 
\hline 
St & \cellcolor{green} & \cellcolor{green} & \cellcolor{green} & \cellcolor{green} & \cellcolor{green} & \cellcolor{green} & \cellcolor{green} & \cellcolor{green} & \cellcolor{green} & \cellcolor{red} \\ \hline 
T1 & & \cellcolor{blue} & & & & \cellcolor{blue} & & & & \\ \hline 
T2 & & & & \cellcolor{purple} & \cellcolor{purple} & & & & \cellcolor{purple} & \cellcolor{purple} \\ \hline 
T3 & \cellcolor{cyan} & & \cellcolor{cyan} & & & & \cellcolor{cyan} & \cellcolor{cyan} & & \\ \hline 
\end{tabular} 
\caption{ Least Laxity First 1 } 
\end{table} 
Escala Bloque : Ciclos = 1 : 1 \\ 
Posicion Fallo:  9 \\ 
mcm:  60 \\ 
=======
\resizebox{!}{.07\linewidth}{ 
\begin{tabular}{|l|l|l|l|l|l|l|l|} 
\hline 
St &  1 \cellcolor{green} &  2 \cellcolor{green} &  3 \cellcolor{green} &  4 \cellcolor{green} &  5 \cellcolor{green} &  6 \cellcolor{green} &  7 \cellcolor{red} \\ \hline 
T1 & \cellcolor{blue} & & & & & & \\ \hline 
T2 & & \cellcolor{purple} & \cellcolor{purple} & & & & \\ \hline 
T3 & & & & \cellcolor{cyan} & \cellcolor{cyan} & \cellcolor{cyan} & \\ \hline 
T4 & & & & & & & \\ \hline 
T5 & & & & & & & \\ \hline 
T6 & & & & & & \\ \hline 
\end{tabular} 
} 
\caption{ Rate Monotonic 1 } 
\end{table} 
\begin{table} 
\centering 
\resizebox{!}{.07\linewidth}{ 
\begin{tabular}{|l|l|l|l|l|l|l|l|} 
\hline 
St &  1 \cellcolor{green} &  2 \cellcolor{green} &  3 \cellcolor{green} &  4 \cellcolor{green} &  5 \cellcolor{green} &  6 \cellcolor{green} &  7 \cellcolor{red} \\ \hline 
T1 & & & & & & & \\ \hline 
T2 & & & & & & & \\ \hline 
T3 & & & & & & & \\ \hline 
T4 & & & & & & & \\ \hline 
T5 & & & & & & & \cellcolor{yellow} \\ \hline 
T6 & \cellcolor{orange} & \cellcolor{orange} & \cellcolor{orange} & \cellcolor{orange} & \cellcolor{orange} & \cellcolor{orange} \\ \hline 
\end{tabular} 
} 
\caption{ Earliest Dead First 1 } 
\end{table} 
\begin{table} 
\centering 
\resizebox{!}{.07\linewidth}{ 
\begin{tabular}{|l|l|l|l|l|l|l|l|} 
\hline 
St &  1 \cellcolor{green} &  2 \cellcolor{green} &  3 \cellcolor{green} &  4 \cellcolor{green} &  5 \cellcolor{green} &  6 \cellcolor{green} &  7 \cellcolor{red} \\ \hline 
T1 & & & & & & & \\ \hline 
T2 & & & & & & & \\ \hline 
T3 & & & & & & & \\ \hline 
T4 & & & & & & \cellcolor{gray} & \\ \hline 
T5 & & & \cellcolor{yellow} & & \cellcolor{yellow} & & \cellcolor{yellow} \\ \hline 
T6 & \cellcolor{orange} & \cellcolor{orange} & & \cellcolor{orange} & & \\ \hline 
\end{tabular} 
} 
\caption{ Least Laxity First 1 } 
\end{table} 
\end{frame} 

%------------------------------------------------
\subsection{Informacion de Tabla de Tiempo Completa} 

%------------------------------------------------
\begin{frame} 
\frametitle{Informacion de Tabla de Tiempo Completa} 
Informacion General:\\ 
\begin{itemize} 
\item Escala Bloque : Ciclos = 1 : 1 \\ 
\item mcm:  6 \\ 
\end{itemize} 
Informacion de Rate Monotonic:\\ 
\begin{itemize} 
\item Posicion Fallo:  7 \\ 
\end{itemize} 
Informacion de Earliest Dead First:\\ 
\begin{itemize} 
\item Posicion Fallo:  7 \\ 
\end{itemize} 
Informacion de Least Laxity First:\\ 
\begin{itemize} 
\item Posicion Fallo:  7 \\ 
\end{itemize} 
>>>>>>> fd36befdf5b77500bc70c153b608ea71d8d72bbf
\end{frame} 

%------------------------------------------------
%----------------------------------------------------------------------------------------
% END OF PRESENTATION SLIDES
%----------------------------------------------------------------------------------------
\end{document} 
