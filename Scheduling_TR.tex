\documentclass[xcolor=table]{beamer}

\mode<presentation> {

%\usetheme{default}
%\usetheme{AnnArbor}
%\usetheme{Antibes}
%\usetheme{Bergen}
%\usetheme{Berkeley}
%\usetheme{Berlin}
%\usetheme{Boadilla}
%\usetheme{CambridgeUS}
%\usetheme{Copenhagen}
%\usetheme{Darmstadt}
%\usetheme{Dresden}
%\usetheme{Frankfurt}
%\usetheme{Goettingen}
%\usetheme{Hannover}
%\usetheme{Ilmenau}
%\usetheme{JuanLesPins}
%\usetheme{Luebeck}
\usetheme{Madrid}
%\usetheme{Malmoe}
%\usetheme{Marburg}
%\usetheme{Montpellier}
%\usetheme{PaloAlto}
%\usetheme{Pittsburgh}
%\usetheme{Rochester}
%\usetheme{Singapore}
%\usetheme{Szeged}
%\usetheme{Warsaw}

%\usecolortheme{albatross}
%\usecolortheme{beaver}
%\usecolortheme{beetle}
%\usecolortheme{crane}
%\usecolortheme{dolphin}
%\usecolortheme{dove}
%\usecolortheme{fly}
%\usecolortheme{lily}
%\usecolortheme{orchid}
%\usecolortheme{rose}
%\usecolortheme{seagull}
%\usecolortheme{seahorse}
%\usecolortheme{whale}
%\usecolortheme{wolverine}

\setbeamertemplate{footline} % To remove the footer line in all slides uncomment this line
%\setbeamertemplate{footline}[page number] % To replace the footer line in all slides with a simple slide count uncomment this line

%\setbeamertemplate{navigation symbols}{} % To remove the navigation symbols from the bottom of all slides uncomment this line
}

\renewcommand{\tablename}{Tabla}

\usepackage{graphicx} % Allows including images
\usepackage{booktabs} % Allows the use of \toprule, \midrule and \bottomrule in tables
\usepackage{xcolor}

%----------------------------------------------------------------------------------------
%	TITLE PAGE
%----------------------------------------------------------------------------------------

\title[Proyecto 3]{Proyecto 3: Scheduling en Tiempo Real} % The short title appears at the bottom of every slide, the full title is only on the title page

\author{Oscar Blandino, Emmanuel Barrantes, Esteban Chavarria} % Your name
\institute[TEC] % Your institution as it will appear on the bottom of every slide, may be shorthand to save space
{
Tecnologico de Costa Rica \\ % Your institution for the title page
Sistemas Operativos Avanzados \\
I Semestre - 2018
\medskip
\textit{} % Your email address
}
\date{\today} % Date, can be changed to a custom date

\begin{document}

\begin{frame}
\titlepage % Print the title page as the first slide
\end{frame}

%\begin{frame}
%\frametitle{Overview} % Table of contents slide, comment this block out to remove it
%\tableofcontents % Throughout your presentation, if you choose to use \section{} and \subsection{} commands, these will automatically be printed on this slide as an overview of your presentation
%\end{frame}

%----------------------------------------------------------------------------------------
% BEGIN OF PRESENTATION SLIDES
%----------------------------------------------------------------------------------------
%------------------------------------------------

\section{Rate Monotonic}

\subsection{Algoritmo RM}

\begin{frame} 
\frametitle{Algoritmo Rate Monotonic (RM)} 

Presentado por Liu y Layland en 1973. \\
Algoritmo de scheduling dinamico, derivado de HPF Prioridades Fijas. \\
Supuestos de RM:
\begin{itemize}
\item Las solicitudes de todas las tareas criticas son periodicas.
\item Todas las tareas son independientes.
\item El deadline de cada tarea es igual al periodo.
\item Se conoce a priori el tiempo de computacion.
\item El cambio de contexto es igual a cero, o esta incluido en el tiempo de computacion.
\end{itemize} 

La condicion suficiente para la schedulability de $n$ tareas bajo RM es:
\begin{center}
$\mu \leq U(n) = n(2^{\frac{1}{n}}-1)$
\end{center}

Donde:
\begin{center}
$\mu = \sum_{i=1}^{n}\frac{c_i}{p_i}$
\end{center}

\end{frame}

%------------------------------------------------
%------------------------------------------------

\section{Earliest Deadline First}

\subsection{Algoritmo EDF}

\begin{frame} 
\frametitle{Algoritmo Earliest Deadline First (EDF)} 

Presentado por Liu y Layland en 1973. \\
Algoritmo de scheduling dinamico, derivado de HPF Prioridades Fijas. \\
Supuestos de EDF son los mismos que los de RM. \\
Prioridad es inversamente proporcional al tiempo pendiente para que se de el deadline de la tarea. \\
Algoritmo expropiativo:
\begin{itemize}
\item Cuando una tarea llega a la cola de ready se revisan las prioridades.
\item Se puede quitar a una tarea que este usando el CPU.
\end{itemize} 


La condicion necesaria y suficiente para la schedulability de $n$ tareas bajo EDF es:
\begin{center}
$\mu \leq 1$
\end{center}

Donde:
\begin{center}
$\mu = \sum_{i=1}^{n}\frac{c_i}{p_i}$
\end{center}

\end{frame}

%------------------------------------------------
%------------------------------------------------

\section{Least Laxity First}

\subsection{Algoritmo LLF}

\begin{frame} 
\frametitle{Algoritmo Least Laxity First (LLF)} 

Prioridad es inversamente proporcional al laxity de la tarea. \\
Algoritmo expropiativo\\
Supuestos de LLF:
\begin{itemize}
\item Las solicitudes de todas las tareas críticas son periódicas.
\item Todas las tareas son independientes.
\item Se conoce a priori el tiempo de computacion.
\item El cambio de contexto es igual a cero, o esta incluido en el tiempo de computacion.
\item El Laxity corresponde a: $X_i = d_i - t - c_i$.
\end{itemize} 

Donde $d_i$ es el deadline, $t$ es el tiempo actual, y $c_i$ es el tiempo de ejecucion que le falta a la tarea.

La condicion necesaria y suficiente para la schedulability de $n$ tareas bajo LLF es:
\begin{center}
$\mu \leq 1$
\end{center}

Donde:
\begin{center}
$\mu = \sum_{i=1}^{n}\frac{c_i}{p_i}$
\end{center}

\end{frame}

%------------------------------------------------
\subsection{Tests de Schedulability  Rate Monotonic } 

%------------------------------------------------
\begin{frame} 
\frametitle{Tests de Schedulability  Rate Monotonic } 
\begin{table} 
\centering 
\begin{tabular}{|l|l|l|} 
\hline 
\cellcolor{lightgray}Tarea & \cellcolor{lightgray}$P_i$ & \cellcolor{lightgray}$C_i$ \\ \hline 
T1   & 9  &  3\\ \hline 
T2   & 8  &  2\\ \hline 
T3   & 7  &  5\\ \hline 
\end{tabular} 
\caption{Datos  Rate Monotonic } 
\end{table} 
Condicion: $\mu \leq U(n)$ \\ 
$\mu =  1,297619 $ \\ 
$U(n) =  0,779763 $ \\ 
Dado que $\mu>U(n)$ el algoritmo indica que las tareas no son schedulable \\ 
\end{frame} 

%------------------------------------------------
\subsection{Tests de Schedulability  Earliest Deadline First } 

%------------------------------------------------
\begin{frame} 
\frametitle{Tests de Schedulability  Earliest Deadline First } 
\begin{table} 
\centering 
\begin{tabular}{|l|l|l|} 
\hline 
\cellcolor{lightgray}Tarea & \cellcolor{lightgray}$P_i$ & \cellcolor{lightgray}$C_i$ \\ \hline 
T1   & 9  &  3\\ \hline 
T2   & 8  &  2\\ \hline 
T3   & 7  &  5\\ \hline 
\end{tabular} 
\caption{Datos  Earliest Deadline First } 
\end{table} 
Condicion: $\mu \leq 1$ \\ 
$\mu =  1,297619 $ \\ 
Dado que $\mu>1$ el algoritmo indica que las tareas no son schedulable \\ 
\end{frame} 

%------------------------------------------------
\subsection{Tests de Schedulability  Least Laxity First } 

%------------------------------------------------
\begin{frame} 
\frametitle{Tests de Schedulability  Least Laxity First } 
\begin{table} 
\centering 
\begin{tabular}{|l|l|l|} 
\hline 
\cellcolor{lightgray}Tarea & \cellcolor{lightgray}$P_i$ & \cellcolor{lightgray}$C_i$ \\ \hline 
T1   & 9  &  3\\ \hline 
T2   & 8  &  2\\ \hline 
T3   & 7  &  5\\ \hline 
\end{tabular} 
\caption{Datos  Least Laxity First } 
\end{table} 
Condicion: $\mu \leq 1$ \\ 
$\mu =  1,297619 $ \\ 
Dado que $\mu>1$ el algoritmo indica que las tareas no son schedulable \\ 
\end{frame} 

%------------------------------------------------
\subsection{Tabla de Tiempo Completa} 

%------------------------------------------------
\begin{frame} 
\frametitle{Tabla de Tiempo Completa} 
\begin{table} 
\centering 
\resizebox{!}{.07\linewidth}{ 
\begin{tabular}{|l|l|l|l|l|l|l|l|l|l|l|} 
\hline 
St &  \cellcolor{green} &  \cellcolor{green} &  \cellcolor{green} &  \cellcolor{green} &  \cellcolor{green} &  \cellcolor{green} &  \cellcolor{green} &  \cellcolor{green} &  \cellcolor{green} &  \cellcolor{red} \\ \hline 
T1 & & & & & & & & & & \\ \hline 
T2 & & & & & & \cellcolor{purple} & \cellcolor{purple} & & & \\ \hline 
T3 & \cellcolor{cyan} & \cellcolor{cyan} & \cellcolor{cyan} & \cellcolor{cyan} & \cellcolor{cyan} & & & \cellcolor{cyan} & \cellcolor{cyan} & \cellcolor{cyan} \\ \hline 
\end{tabular} 
} 
\caption{ Rate Monotonic 1 } 
\end{table} 
\begin{table} 
\centering 
\resizebox{!}{.07\linewidth}{ 
\begin{tabular}{|l|l|l|l|l|l|l|l|l|l|l|} 
\hline 
St &  \cellcolor{green} &  \cellcolor{green} &  \cellcolor{green} &  \cellcolor{green} &  \cellcolor{green} &  \cellcolor{green} &  \cellcolor{green} &  \cellcolor{green} &  \cellcolor{green} &  \cellcolor{red} \\ \hline 
T1 & & & & & & & & \cellcolor{blue} & \cellcolor{blue} & \cellcolor{blue} \\ \hline 
T2 & & & & & & \cellcolor{purple} & \cellcolor{purple} & & & \\ \hline 
T3 & \cellcolor{cyan} & \cellcolor{cyan} & \cellcolor{cyan} & \cellcolor{cyan} & \cellcolor{cyan} & & & & & \\ \hline 
\end{tabular} 
} 
\caption{ Earliest Deadline First 1 } 
\end{table} 
\begin{table} 
\centering 
\resizebox{!}{.07\linewidth}{ 
\begin{tabular}{|l|l|l|l|l|l|l|l|l|l|l|} 
\hline 
St &  \cellcolor{green} &  \cellcolor{green} &  \cellcolor{green} &  \cellcolor{green} &  \cellcolor{green} &  \cellcolor{green} &  \cellcolor{green} &  \cellcolor{green} &  \cellcolor{green} &  \cellcolor{red} \\ \hline 
T1 & & & & & & & \cellcolor{blue} & & \cellcolor{blue} & \\ \hline 
T2 & & & & & & \cellcolor{purple} & & \cellcolor{purple} & & \\ \hline 
T3 & \cellcolor{cyan} & \cellcolor{cyan} & \cellcolor{cyan} & \cellcolor{cyan} & \cellcolor{cyan} & & & & & \cellcolor{cyan} \\ \hline 
\end{tabular} 
} 
\caption{ Least Laxity First 1 } 
\end{table} 
\end{frame} 

%------------------------------------------------
\subsection{Informacion de Tabla de Tiempo Completa} 

%------------------------------------------------
\begin{frame} 
\frametitle{Informacion de Tabla de Tiempo Completa} 
Informacion General:\\ 
\begin{itemize} 
\item Escala Bloque : Ciclos = 1 : 1 \\ 
\item mcm:  504 \\ 
\end{itemize} 
Informacion de Rate Monotonic:\\ 
\begin{itemize} 
\item Posicion Fallo:  10 \\ 
\end{itemize} 
Informacion de Earliest Deadline First:\\ 
\begin{itemize} 
\item Posicion Fallo:  10 \\ 
\end{itemize} 
Informacion de Least Laxity First:\\ 
\begin{itemize} 
\item Posicion Fallo:  10 \\ 
\end{itemize} 
\end{frame} 

%------------------------------------------------
%----------------------------------------------------------------------------------------
% END OF PRESENTATION SLIDES
%----------------------------------------------------------------------------------------
\end{document} 
